\pagestyle{myheadings}
\markboth{ÖNSÖZ}{ÖNSÖZ}

\chapter*{Önsöz [3.5hr]}
Bu kitabın 1986 edisyonundan bu yana derleyici tasarımı dünyası ciddi ölçüde değişti. Programlama dilleri gelişti ve yeni derleme problemleri ortaya çıktı. Bilgisayar mimarileri, bir derleyici tasarımcısının faydalanması gereken çeşitli kaynaklar sunuyor. Büyük ihtimalle en ilginç olanı, saygıdeğer kod optimizasyon teknolojisi derleyiciler dışında da kullanım alanlarına kavuştu. Şimdilerde yazılımlarda hataları bulan programlarda, hatta daha da önemlisi var olan kodda güvenlik problemlerini bulmak üzere kullanılıyor. Ve daha bir çok 'front end' teknoloji -gramerler, düzenli ifadeler, parser'lar ve sentaks-yönelimli dönüştürücüler- hala çokça kullanılıyor.

Dolayısı ile geçmiş edisyonlardan kalma felsefemiz değişmedi. Biliyoruz ki birkaç okuyucu başlıca bir programlama dili için derleyici tasarlayacaktır. Yine de, derleyiciler ile ilişkilendirilen modeller, teori ve algoritmalar yazılım tasarımı ve yazılım geliştirme alanlarındaki geniş bir yelpazedeki problemler için kullanılabilir. Bu yüzden dilin kaynağından veya hedef makina'dan bağımsız olarak dil işlemcisi tasarımında karşılaşılan problemlerin üstüne basıyoruz.

\section*{Kitabın Kullanımı}

Kitaptaki meteryallerin çoğunu işlemek için neredeyse iki Güz dönemi gerekiyor. Yaygın olarak kitabın ilk yarısı üniversiteye girişte gerisi okulun sonlarına doğru işlenir. Genel hatlarıyla bölümleri anlatmak gerekirse;
\\

Bölüm 1 motive edici meteryalleri, ayrıca bilgisayar mimarisi ve programlama dili prensiplerinin arka planında karşılaşılan bazı problemleri içerir.
\\

Bölüm 2 minyatür bir derleyici geliştirip ileride daha da detaylandırılacak bir çok önemli konsepti tanıtır. Derleyicinin kendisi Ek'tedir.
\\

Bölüm 3 lexical analiz, düzenli ifadeler, sonlu-durum makineleri ve tarayıcı-üretici araçları kapsar. Bu meteryaller her türlü metin işlemede temeldir.
\\

Bölüm 4 başlıca \textit{yukarıdan aşağıya} (tekrarlı-azalan, LL) ve \textit{alttan yukarıya} çözümleme metodlarını kapsar
\\

Bölüm 5 sentaks-yönelimli tanımlarda ve dönüştürücülerde başlıca fikirleri sunar.
\\

Bölüm 6 Bölüm 5'in teorisini alıp bunu tipik bir programlama dili için orta seviye kod oluşturmak için nasıl kullanılacağını gösterir.
\\

Bölüm 7 çalışma-anı ortamlarını, özellikle çalışma-anı yığınını ve atık toplamasını kapsar.
\\

Bölüm 8 nesne-kodu üretimi üzerinedir. Temel yapıların işasını, ifadelerden ve ve temel bloklardan kod oluşturulmasını ve register-tahsisi tekniklerini kapsar.
\\

Bölüm 9 akış grafiklerini, veri-akış framework'larını ve bu framework'ları çözmek için tekrarlı algoritmaları içeren kod optimizasyonu teknolojisini tanıtır.
\\

Bölüm 10 komut-seviyesi optimizasyonunu kapsar. Üzerinde durulan konu, paralelizm'in küçük komut dizelerinden çıkarılması ve bunları aynı anda birden fazla işlem yapabilen işlemciler üzerinde programlanmasıdır.
\\

Bölüm 11 geniş-ölçekli paralelizm tespiti ve bundan faydalanılması hakkında tartışır. Üzerinde durulan konu, çok boyutlu dizelere yayılan sık döngülere sahip nümerik kodlardır.
\\

Bölüm 12 prosedürlerarası analiz üzerinedir. Pointer analizini, örtüşmeyi ve kod içinde verilen bir noktaya uzanan prosedür çağrıları dizisini de içeren veri-akış analizlerini kapsar.
\\

Bu meteryaller ile Columbia, Harvard ve Stanford üniversitesinde dersler işlendi. Columbia'da ilk yıl programlama dilleri ve dönüştürücüler dersi sıklıkla ilk 8 bölümden meteryalleri kullandı/önerdi. Dersin amacı küçük gruplar halinde öğrencilerin kendi dillerini tasarladıkları bir Güz dönemi projesidir. Öğrenci yaratımı olan bu projeler çeşitli uygulama alanlarını -kuantum bilişim, müzik sentezi, bilgisayar grafiği, oyun vb.- kapsıyor. Öğrenciler Bölüm 2 ve 5'te bahsedilen ANTLR, Lex ve Yacc gibi derleyici-komponent üreticilerini ve ayrıca sentaks-yönelimli çevrim tekniklerini kullanarak kendi derleyicilerini tasarlayabiliyorlar. Bunu takip eden derste ağ işlemcileri ve çok işlemcili mimariler gibi modern makineler için kod oluşturma ve optimizasyonuna vurgu yapan   Bölüm 9'dan 12'ye kadar olan kısma odaklanırlar.

Stanford'da çeyrek dönem giriş niteliğinde bir ders kabaca Bölüm 1'den 8'e kadar olan içerikleri kapsar, fakat Bölüm 9'dan global kod optimizasyonu konusu da işlenir. Derleyiciler üzerinde ikinci bir ders Bölüm 9'dan 12'ye kadar ve Bölüm 7'den ileri seviye atık toplayıcı içerikleri kapsar. Öğrenciler kurum içi geliştirilmiş, veri-akışı analiz algoritmalarını uygulamak için {\fontfamily{ptm}\selectfont 
Joeq
}adlı Java-tabanlı bir sistem kullanırlar.

\section*{Ön Koşullar}
Okuyucu ikinci sınıf programlama, veri yapıları ve ayrık matematik olmak üzere biraz bilgisayar bilimleri bilgisine sahip olmalıdır. Bir kaç farklı programlama dili bilgisi de yararlı olur.

\section*{Alıştırmalar}
Kitap, neredeyse her bölümde geniş kapsamlı alıştırmalar içerir. Zorlu olan alıştırmaları veya alıştırma parçalarını ünlem işareti, daha zorlu olanları ise çift ünlem işareti ile belirteceğiz.

\section*{Teşekkür}
Jon Bentley bu kitabın taslağı üzerinde birden fazla bölümde bize geniş çaplı yorumlarda bulundu. Bize yardımcı olan yorumlar ve hatalar aşağıda sayacağımız insanlar tarafından bildirildi;
-

Yardımları için bu insanlara teşekkürlerimizi sunuyoruz. Hala kitapta bulunan hatalar tabiki bize aittir.

Ayrıca Monica, 18-yıllık bir eğitim için SUIF'in derleyici takımındaki iş arkadaşlarına teşekkür etmek istiyor;